% Options for packages loaded elsewhere
\PassOptionsToPackage{unicode}{hyperref}
\PassOptionsToPackage{hyphens}{url}
\PassOptionsToPackage{dvipsnames,svgnames,x11names}{xcolor}
%
\documentclass[
  letterpaper,
  DIV=11,
  numbers=noendperiod]{scrartcl}

\usepackage{amsmath,amssymb}
\usepackage{lmodern}
\usepackage{iftex}
\ifPDFTeX
  \usepackage[T1]{fontenc}
  \usepackage[utf8]{inputenc}
  \usepackage{textcomp} % provide euro and other symbols
\else % if luatex or xetex
  \usepackage{unicode-math}
  \defaultfontfeatures{Scale=MatchLowercase}
  \defaultfontfeatures[\rmfamily]{Ligatures=TeX,Scale=1}
\fi
% Use upquote if available, for straight quotes in verbatim environments
\IfFileExists{upquote.sty}{\usepackage{upquote}}{}
\IfFileExists{microtype.sty}{% use microtype if available
  \usepackage[]{microtype}
  \UseMicrotypeSet[protrusion]{basicmath} % disable protrusion for tt fonts
}{}
\makeatletter
\@ifundefined{KOMAClassName}{% if non-KOMA class
  \IfFileExists{parskip.sty}{%
    \usepackage{parskip}
  }{% else
    \setlength{\parindent}{0pt}
    \setlength{\parskip}{6pt plus 2pt minus 1pt}}
}{% if KOMA class
  \KOMAoptions{parskip=half}}
\makeatother
\usepackage{xcolor}
\setlength{\emergencystretch}{3em} % prevent overfull lines
\setcounter{secnumdepth}{-\maxdimen} % remove section numbering
% Make \paragraph and \subparagraph free-standing
\ifx\paragraph\undefined\else
  \let\oldparagraph\paragraph
  \renewcommand{\paragraph}[1]{\oldparagraph{#1}\mbox{}}
\fi
\ifx\subparagraph\undefined\else
  \let\oldsubparagraph\subparagraph
  \renewcommand{\subparagraph}[1]{\oldsubparagraph{#1}\mbox{}}
\fi

\usepackage{color}
\usepackage{fancyvrb}
\newcommand{\VerbBar}{|}
\newcommand{\VERB}{\Verb[commandchars=\\\{\}]}
\DefineVerbatimEnvironment{Highlighting}{Verbatim}{commandchars=\\\{\}}
% Add ',fontsize=\small' for more characters per line
\usepackage{framed}
\definecolor{shadecolor}{RGB}{241,243,245}
\newenvironment{Shaded}{\begin{snugshade}}{\end{snugshade}}
\newcommand{\AlertTok}[1]{\textcolor[rgb]{0.68,0.00,0.00}{#1}}
\newcommand{\AnnotationTok}[1]{\textcolor[rgb]{0.37,0.37,0.37}{#1}}
\newcommand{\AttributeTok}[1]{\textcolor[rgb]{0.40,0.45,0.13}{#1}}
\newcommand{\BaseNTok}[1]{\textcolor[rgb]{0.68,0.00,0.00}{#1}}
\newcommand{\BuiltInTok}[1]{\textcolor[rgb]{0.00,0.23,0.31}{#1}}
\newcommand{\CharTok}[1]{\textcolor[rgb]{0.13,0.47,0.30}{#1}}
\newcommand{\CommentTok}[1]{\textcolor[rgb]{0.37,0.37,0.37}{#1}}
\newcommand{\CommentVarTok}[1]{\textcolor[rgb]{0.37,0.37,0.37}{\textit{#1}}}
\newcommand{\ConstantTok}[1]{\textcolor[rgb]{0.56,0.35,0.01}{#1}}
\newcommand{\ControlFlowTok}[1]{\textcolor[rgb]{0.00,0.23,0.31}{#1}}
\newcommand{\DataTypeTok}[1]{\textcolor[rgb]{0.68,0.00,0.00}{#1}}
\newcommand{\DecValTok}[1]{\textcolor[rgb]{0.68,0.00,0.00}{#1}}
\newcommand{\DocumentationTok}[1]{\textcolor[rgb]{0.37,0.37,0.37}{\textit{#1}}}
\newcommand{\ErrorTok}[1]{\textcolor[rgb]{0.68,0.00,0.00}{#1}}
\newcommand{\ExtensionTok}[1]{\textcolor[rgb]{0.00,0.23,0.31}{#1}}
\newcommand{\FloatTok}[1]{\textcolor[rgb]{0.68,0.00,0.00}{#1}}
\newcommand{\FunctionTok}[1]{\textcolor[rgb]{0.28,0.35,0.67}{#1}}
\newcommand{\ImportTok}[1]{\textcolor[rgb]{0.00,0.46,0.62}{#1}}
\newcommand{\InformationTok}[1]{\textcolor[rgb]{0.37,0.37,0.37}{#1}}
\newcommand{\KeywordTok}[1]{\textcolor[rgb]{0.00,0.23,0.31}{#1}}
\newcommand{\NormalTok}[1]{\textcolor[rgb]{0.00,0.23,0.31}{#1}}
\newcommand{\OperatorTok}[1]{\textcolor[rgb]{0.37,0.37,0.37}{#1}}
\newcommand{\OtherTok}[1]{\textcolor[rgb]{0.00,0.23,0.31}{#1}}
\newcommand{\PreprocessorTok}[1]{\textcolor[rgb]{0.68,0.00,0.00}{#1}}
\newcommand{\RegionMarkerTok}[1]{\textcolor[rgb]{0.00,0.23,0.31}{#1}}
\newcommand{\SpecialCharTok}[1]{\textcolor[rgb]{0.37,0.37,0.37}{#1}}
\newcommand{\SpecialStringTok}[1]{\textcolor[rgb]{0.13,0.47,0.30}{#1}}
\newcommand{\StringTok}[1]{\textcolor[rgb]{0.13,0.47,0.30}{#1}}
\newcommand{\VariableTok}[1]{\textcolor[rgb]{0.07,0.07,0.07}{#1}}
\newcommand{\VerbatimStringTok}[1]{\textcolor[rgb]{0.13,0.47,0.30}{#1}}
\newcommand{\WarningTok}[1]{\textcolor[rgb]{0.37,0.37,0.37}{\textit{#1}}}

\providecommand{\tightlist}{%
  \setlength{\itemsep}{0pt}\setlength{\parskip}{0pt}}\usepackage{longtable,booktabs,array}
\usepackage{calc} % for calculating minipage widths
% Correct order of tables after \paragraph or \subparagraph
\usepackage{etoolbox}
\makeatletter
\patchcmd\longtable{\par}{\if@noskipsec\mbox{}\fi\par}{}{}
\makeatother
% Allow footnotes in longtable head/foot
\IfFileExists{footnotehyper.sty}{\usepackage{footnotehyper}}{\usepackage{footnote}}
\makesavenoteenv{longtable}
\usepackage{graphicx}
\makeatletter
\def\maxwidth{\ifdim\Gin@nat@width>\linewidth\linewidth\else\Gin@nat@width\fi}
\def\maxheight{\ifdim\Gin@nat@height>\textheight\textheight\else\Gin@nat@height\fi}
\makeatother
% Scale images if necessary, so that they will not overflow the page
% margins by default, and it is still possible to overwrite the defaults
% using explicit options in \includegraphics[width, height, ...]{}
\setkeys{Gin}{width=\maxwidth,height=\maxheight,keepaspectratio}
% Set default figure placement to htbp
\makeatletter
\def\fps@figure{htbp}
\makeatother

\KOMAoption{captions}{tableheading}
\makeatletter
\makeatother
\makeatletter
\makeatother
\makeatletter
\@ifpackageloaded{caption}{}{\usepackage{caption}}
\AtBeginDocument{%
\ifdefined\contentsname
  \renewcommand*\contentsname{Table of contents}
\else
  \newcommand\contentsname{Table of contents}
\fi
\ifdefined\listfigurename
  \renewcommand*\listfigurename{List of Figures}
\else
  \newcommand\listfigurename{List of Figures}
\fi
\ifdefined\listtablename
  \renewcommand*\listtablename{List of Tables}
\else
  \newcommand\listtablename{List of Tables}
\fi
\ifdefined\figurename
  \renewcommand*\figurename{Figure}
\else
  \newcommand\figurename{Figure}
\fi
\ifdefined\tablename
  \renewcommand*\tablename{Table}
\else
  \newcommand\tablename{Table}
\fi
}
\@ifpackageloaded{float}{}{\usepackage{float}}
\floatstyle{ruled}
\@ifundefined{c@chapter}{\newfloat{codelisting}{h}{lop}}{\newfloat{codelisting}{h}{lop}[chapter]}
\floatname{codelisting}{Listing}
\newcommand*\listoflistings{\listof{codelisting}{List of Listings}}
\makeatother
\makeatletter
\@ifpackageloaded{caption}{}{\usepackage{caption}}
\@ifpackageloaded{subcaption}{}{\usepackage{subcaption}}
\makeatother
\makeatletter
\@ifpackageloaded{tcolorbox}{}{\usepackage[many]{tcolorbox}}
\makeatother
\makeatletter
\@ifundefined{shadecolor}{\definecolor{shadecolor}{rgb}{.97, .97, .97}}
\makeatother
\makeatletter
\makeatother
\ifLuaTeX
  \usepackage{selnolig}  % disable illegal ligatures
\fi
\IfFileExists{bookmark.sty}{\usepackage{bookmark}}{\usepackage{hyperref}}
\IfFileExists{xurl.sty}{\usepackage{xurl}}{} % add URL line breaks if available
\urlstyle{same} % disable monospaced font for URLs
\hypersetup{
  pdftitle={Untitled},
  colorlinks=true,
  linkcolor={blue},
  filecolor={Maroon},
  citecolor={Blue},
  urlcolor={Blue},
  pdfcreator={LaTeX via pandoc}}

\title{Untitled}
\author{}
\date{}

\begin{document}
\maketitle
\ifdefined\Shaded\renewenvironment{Shaded}{\begin{tcolorbox}[interior hidden, enhanced, breakable, frame hidden, borderline west={3pt}{0pt}{shadecolor}, sharp corners, boxrule=0pt]}{\end{tcolorbox}}\fi

\hypertarget{clean-data}{%
\section{Clean data}\label{clean-data}}

First we need to remove race.

\begin{Shaded}
\begin{Highlighting}[]
\ImportTok{import}\NormalTok{ pandas }\ImportTok{as}\NormalTok{ pd}
\NormalTok{df }\OperatorTok{=}\NormalTok{ pd.read\_csv(}\StringTok{\textquotesingle{}lending\_train.csv\textquotesingle{}}\NormalTok{,index\_col}\OperatorTok{=}\DecValTok{0}\NormalTok{)}
\NormalTok{df }\OperatorTok{=}\NormalTok{ df.drop(columns}\OperatorTok{=}\StringTok{\textquotesingle{}race\textquotesingle{}}\NormalTok{)}
\NormalTok{display(df.head(}\DecValTok{3}\NormalTok{).iloc[:}\DecValTok{4}\NormalTok{])}
\NormalTok{display(df.head(}\DecValTok{3}\NormalTok{).iloc[}\DecValTok{4}\NormalTok{:])}
\end{Highlighting}
\end{Shaded}

\begin{verbatim}
c:\Users\sean1\Documents\4Y\CS 490\capstone-bank-lending-prediction\venv\Lib\site-packages\IPython\core\formatters.py:342: FutureWarning:

In future versions `DataFrame.to_latex` is expected to utilise the base implementation of `Styler.to_latex` for formatting and rendering. The arguments signature may therefore change. It is recommended instead to use `DataFrame.style.to_latex` which also contains additional functionality.
\end{verbatim}

\begin{tabular}{lrlllllrrlrlllrrrrrrrlrr}
\toprule
{} &  requested\_amnt & loan\_duration &                      employment & employment\_length &     reason\_for\_loan &     extended\_reason &  annual\_income &  debt\_to\_income\_ratio & employment\_verified &  public\_bankruptcies & zipcode & state & home\_ownership\_status &  delinquency\_last\_2yrs &  fico\_score\_range\_low &  fico\_score\_range\_high &  fico\_inquired\_last\_6mths &  months\_since\_last\_delinq &  revolving\_balance &  total\_revolving\_limit & type\_of\_application &  any\_tax\_liens &  loan\_paid \\
ID &                 &               &                                 &                   &                     &                     &                &                       &                     &                      &         &       &                       &                        &                       &                        &                           &                           &                    &                        &                     &                &            \\
\midrule
0  &         32000.0 &     60 months &                             SVP &           4 years &  debt\_consolidation &  Debt consolidation &       250000.0 &                 16.35 &            Verified &                  0.0 &   333xx &    FL &                  RENT &                    0.0 &                 775.0 &                  779.0 &                       0.0 &                       NaN &            22480.0 &               105700.0 &          Individual &            0.0 &          1 \\
1  &          6000.0 &     36 months &    METAL FABRICATION SUPERVISOR &         10+ years &    home\_improvement &    Home improvement &        70000.0 &                 15.22 &     Source Verified &                  0.0 &   727xx &    AR &              MORTGAGE &                    0.0 &                 650.0 &                  654.0 &                       1.0 &                      30.0 &             6313.0 &                14600.0 &          Individual &            0.0 &          1 \\
2  &          4200.0 &     36 months &  insurance collector specialist &         10+ years &  debt\_consolidation &  Debt consolidation &        37000.0 &                 20.99 &        Not Verified &                  0.0 &   146xx &    NY &                   OWN &                    1.0 &                 665.0 &                  669.0 &                       0.0 &                      11.0 &             4482.0 &                10300.0 &          Individual &            0.0 &          1 \\
\bottomrule
\end{tabular}

\begin{verbatim}
c:\Users\sean1\Documents\4Y\CS 490\capstone-bank-lending-prediction\venv\Lib\site-packages\IPython\core\formatters.py:342: FutureWarning:

In future versions `DataFrame.to_latex` is expected to utilise the base implementation of `Styler.to_latex` for formatting and rendering. The arguments signature may therefore change. It is recommended instead to use `DataFrame.style.to_latex` which also contains additional functionality.
\end{verbatim}

\begin{tabular}{lrlllllrrlrlllrrrrrrrlrr}
\toprule
Empty DataFrame
Columns: Index(['requested\_amnt', 'loan\_duration', 'employment', 'employment\_length',
       'reason\_for\_loan', 'extended\_reason', 'annual\_income',
       'debt\_to\_income\_ratio', 'employment\_verified', 'public\_bankruptcies',
       'zipcode', 'state', 'home\_ownership\_status', 'delinquency\_last\_2yrs',
       'fico\_score\_range\_low', 'fico\_score\_range\_high',
       'fico\_inquired\_last\_6mths', 'months\_since\_last\_delinq',
       'revolving\_balance', 'total\_revolving\_limit', 'type\_of\_application',
       'any\_tax\_liens', 'loan\_paid'],
      dtype='object')
Index: Int64Index([], dtype='int64', name='ID') \\
\bottomrule
\end{tabular}

\begin{Shaded}
\begin{Highlighting}[]
\BuiltInTok{print}\NormalTok{(}\StringTok{\textquotesingle{}hello\textquotesingle{}}\NormalTok{)}
\end{Highlighting}
\end{Shaded}

\begin{verbatim}
hello
\end{verbatim}



\end{document}
